%\documentclass[twocolumn,showpacs,preprintnumbers,amsmath,amssymb,aps,prb]{revtex4-1}

\documentclass[preprint,showpacs,preprintnumbers,amsmath,amssymb,aps,prb]{revtex4-1}

%\usepackage[colorlinks, allcolors=blue]{hyperref}

\usepackage{amsmath}
\usepackage{graphicx}
\usepackage{amsthm}
\theoremstyle{remark}
\newtheorem{problem}{Problem}

\begin{document}


\title{Molecular dynamics simulation of synchronization of a  driven particle}
 
\author{Tiare Guerrero}
\email{guer9330@pacificu.edu}
\affiliation{Department of Physics, Pacific University, Forest Grove, Oregon 97116}

\author{Danielle McDermott}
\email{mcdermott@pacific.edu}
\affiliation{Department of Physics, Pacific University, Forest Grove, Oregon 97116}

\date{\today}

\begin{abstract}
  Synchronization
  plays an important role in many physical processes.
  We discuss synchronization in a 
  molecular dynamics simulation
  of a single particle
  moving through
  a viscous liquid
  while being driven 
  across a washboard potential energy landscape.
  Our results show many dynamical patterns
  as the landscape and driving force are altered.
  For certain conditions,
  the particle's velocity and location
  are synchronized or 
  phase-locked and
  form closed orbits in phase space.
  Quasi-periodic motion is common, 
  for which the
  dynamical center of motion shifts the
  phase space orbit.
  %DM removed first sentence,
  %which was strong and clear, but isolated in the abstract
  %put those ideas in this final sentence, making it a mouthful [26 words]
  By 
  isolating
  synchronized motion
  in 
  simulations and table-top experiments,
  we 
  can study %DM replaced isolate with study
  complex natural behaviors
  important to many physical processes.
 %[suggest omitting this sentence. Code not in text. We include molecular dynamics code
 % to simulate and characterize the
  %particle dynamics.] 
\end{abstract}

\maketitle 

\section{Introduction} 
Synchronization is a universal phenomena
in which
individual oscillators change frequency due
to external stimuli.\cite{Pikovsky2003}
The
flickering patterns of
candle flames 
mediated by temperature fluctuations,\cite{Okamoto2016}
the vibrations of singing wine glasses interacting 
through sound waves,\cite{Arane2009}
and metronomes vibrating through a supporting platform\cite{Jia2015}
are examples of in-phase environmentally coupled oscillations. 
Biological systems benefit from cooperative
synchronization --
birds  coordinate their wing flaps
to optimize energy use during flight,\cite{Portugal2014}
frogs alternate croaking patterns,\cite{Aihara2014}
humans clap in time with music,\cite{Tranchant2016}
and 
at a cellular level, 
neurons simultaneously fire in cardiac muscle\cite{MartinHall1999}
and brain tissue.\cite{Singer1999}
External forcing
can cause or regulate synchronization.
%where external forces
%add or remove energy from the system.
For example, 
an electrical pacemaker 
regulates a heart beat and
a pulsed light modifies the
flashing pattern of fireflies.

The model we discuss
%DM add
in this paper 
%DM end
resembles an externally forced pendulum.
%damping
%transition
%%DM - focus more on single particle with environment
A single pendulum 
can swing in time with a %be driven or damped by
driving force
like  %a skateboarder in a half-pipe or
a pushed child on a swing,
and may be damped
by mechanical friction.
%of gravity to return
%to an equilibrium position,
%or  to decreasing amplitude
% or air resistance.
We focus on 
synchronized phase-locking or mode-locking,
which 
first appeared in the scientific literature
in 1665 with 
Huygens'  experiments on the
motions of %synchronized 
pendula in wall-mounted clocks.\cite{Bennett2002}
A locked-mode is an integer frequency ratio
in coupled oscillators. %Clock pendula swing in phase
%when coupled through shared vibrations.
In Huygens' clocks,
the mode-locked pendula were observed to swing at the same rate
in the same direction (1:1 mode) or in
opposite directions ($-1$:1 mode).
Higher modes might be achieved
by altering the pendula lengths --
a 2:1 mode occurs when a simple pendulum
swings with twice the period
of another one quarter the length.
%DM transition sentence
%Many subsequent 
Simulations and laboratory experiments
have revealed 
mode-locked synchronization
in a diverse range of 
oscillation phenomena.
%Single oscillators 
%can demonstrate a variety of synchronized behaviors.
%Colloid particles in optical traps are an excellent
%model system to explore %these 
%the dynamics of mode-locking. %synchronization.


Colloids 
provide an ideal system to 
control and study
phase-locked motions
%are useful for
%understanding synchronization
at a single particle level.\cite{Juniper2015,Juniper2017}
% such as synchronized mode-locking can be studied with
%An advantage of colloid studies
In experiments %DM moved the ``in experiments'' to second sentence
typical colloids are   
plastic spheres suspended in
de-ionized water or silica beads suspended in organic solvent.
Colloids can be trapped
with 
radiation pressure from 
a laser beam.\cite{Ashkin1997} 
A colloid centered in an optical trap is 
uniformly bombarded by photons
and oscillates
about a equilibrium point.
Off-center colloids 
experience a net force
due to uneven photon collisions across
the particle's surface.
Diffraction gratings can be used to create
more complex light environments, 
such as periodic patterns of minima
suitable for synchronization studies.\cite{Grier2003}  
A driven colloid
can demonstrate
synchronized
motion within one trap,
or synchronized hopping between multiple traps.
%DM moved this sentence up for better flow
In the following simulations, we confine colloids  
to move along a periodic pattern of 
minima and maxima defined by a sine wave.


Although many systems oscillate,
trapped colloids are useful for 
providing controlled 
measurements of the
step-by-step dynamics of the particle motion.
Because
colloids are large 
and move slowly,  the positions of the
particles
can be measured in real time 
with an optical camera.\cite{Pertsinidis2001}
Simulations provide complementary
information,
where environmental and driving parameters may be precisely
and quickly modified to explore
the many oscillation patterns.

A colloid driven across a periodic
substrate may 
synchronize its location
to the pattern of the external drive.
%moving back and forth in time with
%the beat, or
%moving between substrate minima.
When 
a constant or 
dc drive is applied,
the particle's velocity is modulated by 
the potential energy landscape exerted on the particle.   
Below some threshold  
the dc force is not strong enough to push the particle
across a potential maximum so the average velocity is zero,
a phenomena referred to as pinning.\cite{Reichhardt2017}
Above the pinning threshold,
a dc driven particle
%a particle subject to a constant
%drive force
increases its speed at a rate proportional
to the external drive.  
When the applied force varies periodically,
an ac drive,
%cause
the particle can hop back and forth across
the landscape minima.
When a particle is driven across a substrate
by combined 
ac and dc forces,
it is possible for the particle to demonstrate 
mode-locking.
In these modes,  
%Due to the combination of external drive
%and substrate minima,
the particle hops at a controlled rate such that 
the average particle velocity is 
%of an  driven particle 
fixed for a range of dc drive forces
with a magnitude proportional to the
landscape period and driving frequency.\cite{Reichhardt2015}
%Many synchronized patterns occur,
%controlled by the substrate pattern and driving force.

In this paper we discuss simulations 
of the mode-locked dynamics
of a confined particle driven over
a washboard shaped potential energy landscape.
The model is presented in Sec.~\ref{sec:MD} and 
our results are summarized in
%including the synchronized motion of a single confined particle
%driven across a periodic landscape in 
Sec.~\ref{sec:results}. In Sec.~\ref{sec:conclusion}
we relate  our results 
to a broad variety of physical systems beyond 
optically trapped colloids.
%and superconducting vortices in the presence of periodic pinning arrays.
We include problems for interested readers 
in Sec.~\ref{sec:problems}.

\section{Simulation}
\label{sec:MD}
We use a classical model for 
studying the dynamics, 
using the net force on a particle to calculate
its trajectory.
The particle is confined in a one-dimensional 
box of length $L=46.\bar{6}\,a_0$,
where $a_0$ is a dimensionless unit of length.
(In the simulations we set $a_0 = 1$.)
The particle 
has
position $\vec{r} = y \hat{y}$ 
and velocity $\vec{v} = d\vec{r}/dt$.
The edges of the system are treated by
periodic boundary conditions,
such that a particle leaving the edge of the system is mapped
back to a position within the simulation boundaries 
by the transformation $y+L \rightarrow y$. 
We show a schematic of the system in Fig.~\ref{fig:1_landscape}(a).
The units of the simulated variables are summarized in Table~\ref{tab:1}.

We confine the particles using a position dependent 
potential energy, called a landscape or substrate.
The landscape is modulated in the $y$-direction
by the periodic function 
 \begin{equation}
   U(y) = U_0 \cos{(2 \pi y / \lambda)},
     \label{eq:ysubstrate}
\end{equation}
where $\lambda=L/N_p$, with $N_p$ equal to the number of periods,
and $U_0$ is a parameter
 that sets the depth of the minima
 with  units of energy $E_0$. 
 We plot this function in 
 Fig.~\ref{fig:1_landscape}
 for $N_p = 3$.  In Fig.~\ref{fig:1_landscape}(a) we show 
 the $x$-$y$ plane with a contour plot of $U(y)$ 
 to illustrate
 the two-dimensional potential energy landscape;
 the maxima are colored gray and the minima are colored white.  
 Because the motion is one-dimensional, the particle only moves along the $y$-direction in the two-dimensional plane. %[xx added this sentence because the reader might be confused that you stated that the particle is confined to one dimension xx] 

The confining force on the particle 
is given by
 \begin{equation}
 \vec{F}_{\ell}(\vec{r}) = - \vec \nabla U(\vec{r}),
 \label{eq:dudr}
 \end{equation}
where $\ell$ denotes the landscape.
 In Fig.~\ref{fig:1_landscape}(b) we plot  
 $U(y)$ to illustrate how the magnitude
 $|\vec{F}_{\ell}|$ is calculated from the particle position $y$.
 
The particle is subject to an external time-dependent driving force
$\vec{F}_{\rm d}(t)$
applied parallel to the $y$-direction.
We model this force as
\begin{equation}
  \vec{F}_{\rm d}(t) = \big [F_{\rm dc} + F_{\rm ac} \sin(\omega t)\big] \hat{y},
    \label{eq:drive}
\end{equation}
with 
a constant component $F_{\rm dc}$
and a time dependent component with amplitude $F_{\rm ac}$
and angular frequency $\omega = 2 \pi f$.

The inertia of the 
%small
particle is reduced by interactions
with the fluid particles.\cite{Purcell1977}
We assume 
colloids are
overdamped
so the particle does not accelerate,
that is, it is suspended in a continuous viscous fluid
that dissipates energy supplied externally. 
Newton's second law for the particle
is simplified
by the assumption that $\vec{a}$ is zero. 
The overdamped equation of motion for
the velocity $\vec{v}$ of 
an isolated particle is
\begin{equation}
  \eta \vec{v} = \vec{F}_{\ell}(\vec{r}) + \vec{F}_{\rm d}(t),
    \label{eq:motion}
\end{equation}
with friction coefficient $\eta = 1$ in units of $v_0 / F_0$.
The term $-\eta \vec{v}$
is a drag force that models
energy dissipation due to the fluid. 
We discuss  models for
a sphere moving through a fluid in 
Problem~\ref{ex:reynolds}. 

The equation of motion  yields the velocity
of a particle at position $\vec{r}$ 
and simulation time $t$. At each time step
we evaluate the net force on the particle as a function of its position
$\vec{r}(t)$
and then integrate
the equation of motion to move the particle
to its updated position.
Because the acceleration is zero,
the integration of the equation of motion
is performed using 
the simple first-order Euler method 
\begin{equation}
  \vec{r}(t+\Delta t) = \vec{r}(t) + \vec{v}(t) \Delta t
    \label{eq:euler}
\end{equation}
for a time step $\Delta t = 0.1\,\tau$.
In 
Problem~\ref{ex:euler}
we describe a program for 
treating differential equations such as Eq.~\eqref{eq:euler}. %[xx note change xx]

\begin{figure} % [h!]
\centering
\includegraphics[width=\columnwidth]{fig1_landscape.pdf}
\caption{
Schematic of the simulation of a single particle
  driven across a washboard potential energy landscape.
  (a) View of the $x$-$y$ plane. 
  The time-dependent applied driving force $\vec{F}_{\rm d}$
  is parallel to the $y$-axis.
  The landscape is shown with
  maxima of the potential energy in gray 
  and minima in white. 
  The particle is 
  subject to the competing forces of the landscape and the applied driving force.
  (b) The potential energy function $U(y)$. 
  The particle in (a) is shown at the same $y$-position in (b).
  The slope of $U(y)$ is the 
  magnitude of the force $\vec{F}_{\ell}$. 
  }
\label{fig:1_landscape}
\end{figure}

\section{Mode-locking of a single particle}

\label{sec:results}
%Here we drive 
%a single particle 
%across the landscape. 
The numerical implementation of the force due to the landscape 
is calculated using Eqs.~\eqref{eq:ysubstrate} and \eqref{eq:dudr} with
\begin{equation}
  \label{eq:force}
  {F_{\ell}}_y(y) = A_{p} \sin{(2 \pi y / \lambda)},
\end{equation}
where the force is scaled by the parameter $A_{p} = 2\pi U_0/\lambda$.
In this section we fix the landscape parameters
to $A_{p} = 0.1\,F_0$ 
with $N_p=20$, 
corresponding to a spatial period $\lambda = 2.3\,a_0$.
The competition between the driving force and the landscape potential
can produce a variety of hopping patterns in the particle motion. 
The relative values of $F_{\rm ac}$, $F_{\rm dc}$, and $A_p$
control the rate and distance a  particle moves 
forward and backward in the landscape.
If $F_{\rm d}(t) > A_p$, a particle can 
overcome the barrier height of the landscape,
and 
the particle hops between minima in the energy landscape.
If the driving frequency is low,
as in Fig.~\ref{fig:2_Fd_vy_time},  
the driven particle 
moves 
in a pattern 
with the same frequency 
as the time-dependent force $F_{\rm d}(t)$,
but is modulated by the landscape period.
We explore changes in the frequency and different values of $F_{\rm ac}$ 
in Problem~\ref{ex:parameters}.
Here we vary $F_{\rm dc}$ 
while holding the remaining parameters fixed.

\begin{table}[h!]
\centering
\caption{Simulation parameters and units with comparable
  experimental values.~\cite{Juniper2015,Juniper2017} The substrate is scaled by our force units, while an experimental landscape is scaled by the Brownian motion of the particles. }
\begin{ruledtabular}
\begin{tabular}{c c c } 
Quantity & Simulation units & Experimental values\\
\hline
length &  $a_0 = 1$ & $ a_0 \sim 1.5 \mu m$\\
energy & $E_0 = 1$ & \\ %$ < E_0 <$ \\
%electric potential & $V(r_{ij}) = E_0/r_{ij}$ \\
%energy & $E_0 = q^2{Z^*}^2/4\pi \epsilon \epsilon_0 a_0$ \\
%dimensionless interaction strength & $q$ \\
%effective colloidal charge & $Z^*$ \\
%solvent dielectric constant & $\epsilon \epsilon_0$\\
force & $F_0 = E_0 / a_0$ & \\ %$<F_0<$\\
time &  $ \tau = \eta a_0 / F_0 $ & $ \tau \sim 3$\,s\\
velocity &  $ v_0 = a_0 / \tau $ &  $v \sim 5\,\mu$m/s \\
%driving period & $T = 100 \tau$ & $T \sim 1.3$ s \\ %frequency 0.75 Hz
%juniper uses water–ethanol mixture
substrate period & $\lambda = 2.3\,a_0$ & $\lambda = 3.5\,\mu$m \\
substrate amplitude & $A_p = 0.1\,F_0$ & $U_0 = 25 k_B T \sim 1\, {\rm J}$ \\ %$U_0 = 0.037 E_0$ 
temperature & $T = 0$  & $T \sim 290\,$K  \\
\end{tabular}
\end{ruledtabular}
\label{tab:1}
\end{table}

In Fig.~\ref{fig:2_Fd_vy_time}(a)
we plot $F_{\rm d}(t)$ 
as a function of time with $F_{\rm dc}=0.07$, 
$F_{\rm ac}=0.07$, and $f=0.01$ cycles per time unit $\tau$.
The  temporal period of the driving force is
$T = 1/f = 100\,\tau$.
In Fig.~\ref{fig:2_Fd_vy_time}(b) 
we show the $y$-position of the particle
as a function of time,
where we have
normalized $y$ by $\lambda$.
The initial particle position is $y = 0$. 
The particle moves
in the positive $y$-direction
through a distance $\Delta y = \lambda$  in the time $T$,
so that  
the average velocity 
$\langle {v}_y \rangle = \lambda f$. 
The inset of Fig.~\ref{fig:2_Fd_vy_time}(b)
shows $y$ 
over one period $100\,\tau < t < 200\,\tau$
with 
the contour plot described
in Fig.~\ref{fig:1_landscape}(a).
The motion is synchronized such that the 
driving force is a maximum when the landscape 
force is minimum,
as shown by the coincidence of the
steep slope of 
$y/\lambda$ 
and the maxima in $F_{\rm d}(t)$.
When $F_{\rm d}(t)$ is large,
the particle moves across the substrate maxima.

\begin{figure}[h]
\centering
\includegraphics[width=\columnwidth]{fig2_Fd_vy_time.pdf}
\caption{(a) The applied driving force $F_{\rm d}(t)$ 
  with  $F_{\rm dc} = 0.07$, $F_{\rm ac} = 0.07$, and $f=0.01$.
  (b) 
  The $y$-position of the  particle
  normalized by the period of the substrate $\lambda$.
  The substrate strength is $A_p=0.1$.
  The inset   shows
  the $y$-position
  through the second period $100\,\tau<t<200\,\tau$
 along with the contour plot depicting
  the landscape potential shown in Fig.~\ref{fig:1_landscape}(a).
  }
\label{fig:2_Fd_vy_time}
\end{figure}

To explore the possible hopping patterns,
we sweep through a range of values of $F_{\rm dc}$ for fixed $F_{\rm ac}$ and $A_p$.
In Fig.~\ref{fig:3_sweep_vyFDC} 
we increase $F_{\rm dc}$ in increments of $0.001\,F_0$,
and 
measure the average velocity $\langle v_y \rangle $ 
as a function of $F_{\rm dc}$.
We also perform the sweep for a non-oscillatory drive $F_{\rm ac} = 0$.
With no oscillating component of the driving force,
  the force-velocity relation increases monotonically 
  above the depinning threshold $F_c$ such that
  \begin{equation}
    \langle v_y \rangle \propto (F_{\rm dc}-F_c)^{-\beta},
  \end{equation}
  where the power $\beta$ varies with the
  type of phase transition and 
  %can
  may 
  be used to identify the universality class.\cite{Reichhardt2017} 
  The critical force $F_c$ is equal to the maximum substrate force
  $A_p$.
  % [xx would be good to follow up on your mention of universality class in a problem. The exponent beta is usually associated with the decrease of the order parameter, not a divergence xx]
  %DM - in my experience, we use beta for this parameter.  For example
  %Dynamic regimes for driven colloidal particles on a periodic substrate at commensurate and incommensurate fillings D McDermott, J Amelang, CJO Reichhardt, C Reichhardt, Physical Review E 88 (6), 062301
 
The addition of an ac drive leads
  to the formation of modes.
  A mode is a periodic pattern of hops
  with a constant average particle velocity $\langle {v}_{y} \rangle$
  over a range of driving forces $F_{\rm dc}$.
  In Fig.~\ref{fig:3_sweep_vyFDC}
  we sweep $F_{\rm dc}$
  with $F_{\rm ac} = 0.07$ and $f=0.01$.
Each value of $F_{\rm dc}$ yields a different pattern of hops
between substrate minima
performed by the particle
due to the landscape confinement.  %xx note change xx]
At 
low values of $F_{\rm dc}$, the average velocity $\langle v_y \rangle$ is zero.
Because 
$A_p$ is large compared to the extrema of $F_{\rm d}(t)$,
the particle oscillates back and forth
in a single minima with no net velocity, an example of 
a 0:0 mode.
At higher values of $F_{\rm dc}$,    
$\langle v_{y} \rangle$ increases in steps of uniform height,
$\langle v_{y} \rangle = n \lambda f$,
where $n$ is an integer.
We observe a mode of $n=1$
for   $0.05 < F_{\rm dc} < 0.08$,
$n=2$ for $0.08 < F_{\rm dc} < 0.11$,
$n=3$ for $0.12 < F_{\rm dc} < 0.13$,
$n=4$ for $0.14 < F_{\rm dc} < 0.155$, and 
$n=5$ for $0.155 < F_{\rm dc} < 0.16$.
Higher modes are not visible.
The step width is nonlinear
and depends on the strength of $F_{\rm ac}$
for this landscape potential.\cite{Reichhardt2000,Juniper2017}
These steps, known as Shapiro steps, can exhibit a variety of
interesting patterns
such as a devil's staircase related to chaotic dynamics.\cite{Bak1986}

\begin{figure}[h]
\centering
\includegraphics[width=\columnwidth]{fig3_sweep_vyFDC.pdf}
\caption{The average particle velocity  $\langle v_{y} \rangle$
  as a function of $F_{\rm dc}$.
  We let
  $F_{\rm ac}=0.0$ (blue dashed) and 
  $F_{\rm ac}=0.07$ (orange) with $f = 0.01$ 
  as in Fig.~\ref{fig:2_Fd_vy_time}.
  The value of $F_{\rm dc}$ for the first four steps
  corresponds to the fixed value of $F_{\rm dc}$
  in each of the phase plots
  in Figs.~\ref{fig:4_phase}(a)--(d).
}
\label{fig:3_sweep_vyFDC}
\end{figure}

  To study synchronization patterns, 
  it is useful to compare
  mode-locked quantities 
  in a two-dimensional phase plot. 
  For a driven pendulum confined to a single potential well,
  an appropriate
  phase space is the particle velocity $v_y$ versus the position $y$.  
  For a particle driven 
  through multiple identical wells 
  we define phase variables 
  to account for the net increase in the position.
  The phase position is
  \begin{equation}
    \phi(t) = 2\pi [y(t)-\langle v_y \rangle t]/\lambda,
  \end{equation}
  centered about the average particle displacement $\langle v_y \rangle t$
  and normalized by the substrate period $\lambda$.\cite{Juniper2015}
  The phase velocity is
  \begin{equation}
    \dot{\phi}(t) =2\pi [v_y(t)-\langle v_y \rangle] /\lambda.  
  \end{equation}
  %DM removed
  %For a zero landscape force, 
  %the phase velocity  is zero when $F_{\rm d}(t) = F_{\rm dc}$.
If the system
  is strictly mode-locked,
  the particle velocity
  recurs at a particular spatial location, and 
a closed loop appears in  
  phase space. 
  A 1:1 mode appears as a circle or oval. 
  Nodes appear 
  for higher modes,
  sometimes forming figure-eights
  or other recognizable patterns.
  A system that is nearly phase locked
  will appear as an almost closed loop.
  Such quasiperiodic systems are
  not fully synchronized
  so the position-velocity relation
  shifts in time as in Fig.~\ref{fig:4_phase}.
  
  In Fig.~\ref{fig:4_phase}
  we plot $\dot{\phi}(t)$ versus $\phi(t)$
  for increasing values of
  $F_{\rm dc}$, 
  with the remaining parameters fixed as in Fig.~\ref{fig:2_Fd_vy_time}.
  We choose values of $F_{\rm dc}$
  corresponding to the first four modes in
  Fig.~\ref{fig:3_sweep_vyFDC},
  as marked with black circles.
  For
  $F_{\rm dc} = 0.04$
  the phase plot   in Fig.~\ref{fig:4_phase}(a)  is an
  asymmetric curve.
  A tail appears
  due to the initial transient
  motion of the particle.
  The particle is confined to a single
  substrate minima,
  and has no net velocity.
  The asymmetry is caused by the bias induced by $F_{\rm dc}$.
  In Fig.~\ref{fig:4_phase}(b)
  with $F_{\rm dc} = 0.07$
  the phase loop is a symmetric triangular shape,
  indicating a 1:1 match between the
  particle motion and velocity consistent with 
  Fig.~\ref{fig:2_Fd_vy_time}(a).
  As $F_{\rm dc}$ is
  increased,
  nodes form in the phase diagram, which occur
  due to repeated values
  of $\dot{\phi}$ over multiple phase positions.
  In Fig.~\ref{fig:4_phase}(c)
  with $F_{\rm dc} = 0.1$
  two nodes form.
  The particle moves a distance $2\lambda$
  during one time period.
For $F_{\rm dc} = 0.125$ as in Fig.~\ref{fig:4_phase}(d),
three nodes form as the particle moves across $3\lambda$.
%DM add
It is possible to create
more patterns in phase diagrams 
by changing the driving parameters.
At higher values of $F_{\rm dc}$
additional nodes appear.
When the values of $F_{\rm ac}$ and $f$
are tuned to produce backward hops in the substrate,
as suggested in Problem~\ref{ex:parameters}, 
negative nodes appear in $\phi_y(t)$.
%DM end add
  
    \begin{figure}[h!]
      \centering
      \includegraphics[width=\columnwidth]{fig4_phase.pdf}
      \caption{
        Phase plot of $\dot{\phi}(t)/(2\pi)$ versus $\phi(t)/(2\pi)$.
        The particle is driven with $F_{\rm dc}$ equal to  (a) $0.04$, (b) $0.07$, (c) $0.1$, and (d) $0.125$.  These values are denoted as black circles
        in Fig.~\ref{fig:3_sweep_vyFDC}. 
      The other parameters
      are $F_{\rm ac} = 0.07$, $f=0.01$, and $A_p = 0.1$
      as in Fig.~\ref{fig:2_Fd_vy_time}.}
      \label{fig:4_phase}
    \end{figure}

\section{Discussion} %[xx note change
\label{sec:conclusion}	

A single particle driven across a periodic potential landscape 
can synchronize its motion 
to environmental and external forces. 
The simulations we have described 
reproduce the experiments and simulations 
of dynamical 
mode locking in
driven colloids on a
periodic optical landscape.\cite{Juniper2015, Juniper2017}
The simulations can be extended
to include multiple particles
and more complex environments,
leading to
synchronization effects relevant 
to a broad range of experimental systems.
%The model
%is easy to simulate yet relevant
%to a broad range of condensed matter systems.

The study of the motions of particles 
sliding across periodic and aperiodic surfaces 
is relevant 
to a broad range of condensed matter systems.
%beyond optically confined colloids.
Colloids are 
relatively easy to 
manipulate and image in experiments,
making ideal proxies 
for systems
more difficult to access experimentally 
such as cold atoms or electron gases.\cite{Grier2003}
Many systems can be modeled
as particles moving across a surface
including 
the propagation of patterns
in charge and spin properties of atomic systems.
Dynamical mode-locking 
is 
observed in  
quantum electronic
devices as 
stepped regions in the relation between current and voltage,
where the voltage is the analog of the external driving force
and current is the analog of particle velocity.
These mode-locked or phase-locked currents, known as Shapiro steps,
were first
observed in single Josephson junctions.\cite{Shapiro1963, Golubov2004}
Shapiro steps vary in width depending on the strength of the
applied ac forces,
and are observed in a variety of ac and dc driven systems
displaying
non-Ohmic behavior in voltage-current curves.
The study of 
mode-locking in colloids provides insights
into 
%is a useful probe of
complex %quantum mechanical
systems in which 
the motions of individual particles can  be inferred only
from other measurements, 
%insert final sentence summarizing all the things.
including
charge density waves, spin density waves
and superconducting vortices in landscapes 
engineered with periodic patterns of pinning sites.\cite{Reichhardt2000}

\section{Suggested problems}
\label{sec:problems}	

In the following
we explore the behavior of our model
with suggested problems for interested readers.
We discuss the molecular dynamics (MD) algorithm
and
numerical integration techniques in Problem~\ref{ex:euler}.
Changes to the parameters are described in 
Problem~\ref{ex:parameters}.
We include analytic solutions to  the 
linear drag equation in Problem~\ref{ex:reynolds}, and consider
the equation of motion in Problem~\ref{ex:n2l}.
We extend the numerical model  
to include finite temperature effects
in Problem~\ref{ex:brownian}.

\begin{problem}{Write your own MD code}
  \label{ex:euler}
 
  \noindent To calculate the position of the particle
  as a function of time,
  we
  numerically integrate the equation of motion,
  Eq.~(\ref{eq:motion}),
  using
  the  definition of velocity
  $\vec{v} = d\vec{r}/dt$ 
  using the 
  Euler method.
  Equation~(\ref{eq:motion}) provides
  a direct calculation for the particle velocity
  from the net force on the particle,
  as demonstrated in Problem~\ref{ex:n2l}.

    A working example of this code is available
  in Ref.~\onlinecite{supplemental}.
  If you would like to write your own code,
  the following guide
  will walk you through our choices.
  In the code excerpts,
  most comments and optional details are removed for clarity.
We use the Python 
  language
  for educational purposes.
  To model many particles,
we recommend a   
  compiled  language
  to perform the many operations associated
  with neighbor lists to compute particle interactions.
  
  \begin{enumerate}
    
  \item[(a)] {\bf Initialization.}
    For convenience,
    we define a Python dictionary
    to contain the simulation constants
    that could easily be passed by reference
    to subroutines.
 Define 
    the remaining parameters of the system using the dictionary outlined below.
    \begin{verbatim}
def set_parameters():           
    ```set simulation parameters...'''
    #declare the dictionary     
    dict={}  
    dict[`dt'] = 0.1 #time step in simulation units
    #control the oscillating component of driving force
    dict[`F_AC'] = 0.07 #amplitude of force oscillation
    dict[`freq'] = 0.01 #frequency of force oscillation
    #...                #left as an exercise to the reader
    return dict
    \end{verbatim}

    The function is called at the top of the main function
    followed by a call to a subroutine
    containing the  MD algorithm.
     \begin{verbatim}
    if __name__ == "__main__":
        parameters = set_parameters()
        #run the MD simulation
        single_particle(parameters)
\end{verbatim}

   \item[(b)] {\bf Time loop.}
     The Euler method is effective
     for solving linear ordinary differential  equations
     of the form
     $dy/dt = f(t,y(t))$ with the initial condition $y(t_0) = y_0$.
     The solution is calculated 
     by stepping in time through $n$ integer steps
     $t_n = t_0 + n \Delta t$.
     At each subsequent step the new
     value for $y$ is calculated as a  solution of a map using
     discrete times 
     $y_{n+1} = y_n + f(t_n,y_n)$.
     Apply the Euler method to 
     Eq.~(\ref{eq:motion})
     to determine  the analytic expression
     for the position of a particle
     $y_n$ at the $n$th time step.
  
 \qquad Add a {\tt for} loop
     to step through each integer time step.
     This loop 
     controls
     the flow of the program
     and retains information for the particle position and other
     properties
     as a function of time.
     Assume this information
     will be calculated in a subroutine.
        
\qquad We used the subroutine {\tt single\_particle()}
     to calculate the  lengths of the arrays that contain the 
     position, velocity, and time data.
     The lengths of these arrays are  affected by
     how much data you choose to save
     (see comment following sample code).
     Define a loop to 
     calculate the particle position and velocity through each
       time step,
     calculated in the subroutine {\tt md\_step()}
     which will be described in more detail in (c).
\begin{verbatim}
def single_particle(parameters,plot="y-position"):
    ```Run MD simulation...'''    
    #define empty arrays to hold data as a function of time
    #(left as an exercise to the reader)   
    #loop through the integer time steps in the simulation
    for int_time in range(0,maxtime):
        #(left as an exercise to the reader)
        time += dt
\end{verbatim}

{\it Comment:}     
     A key decision for any MD algorithm
     is how much information to save during and after
     the simulation.
     %DM if going to emphasize this, have the reader do something with it
     %cut out - and make a comment rather than a statement going nowhere
     We define the following constants
     to manage the length of arrays
     containing data.
     We found in practice that for 
     short simulations times
     we could save all the data.
\begin{verbatim}
dict[`maxtime']=int(40/dict['freq'])   #total time steps 
dict[`writemovietime']=1   #write data to arrays for plotting
\end{verbatim}

\item[(c)] {\bf Position and force calculations.}
  We created a subroutine to consider each
  type of force in the system (external drive, landscape, etc).
  If the particle moves beyond the limits
  $0 \le y < L$,
  it must be returned 
  using the   periodic boundary conditions discussed
  in Sec.~\ref{sec:MD}.
\begin{verbatim}
  def md_step(y, int_time, avg_vy, parameters, ft=0):
    ```Calculate net force and integrate eq. of motion...'''
    #calculate the floating point value for time
    time = int_time * dt
    #reset vy for every time step because ay = 0 
    vy = 0 
    #calculate the net force on the particle
    vy = #(left as an exercise to the reader)
    #calculate the new position
    y += #(left as an exercise to the reader)
    #check periodic boundary conditions
    #(left as an exercise to the reader)
    return y, vy, avg_vy
\end{verbatim}

  \end{enumerate}

  The Euler algorithm can be applied to
  calculate reasonable numerical solutions to 
  nonlinear
  equations if the time step $\Delta t$
  is kept sufficiently small.\cite{Newman}
  In our simulations we use the time step $\Delta t = 0.1$
  and find no change in the solution
  when we decrease the time step to smaller values.
  In simulations of
  many interacting particles,
  a smaller
  time step is essential for accurate results.
  Particle-particle interactions are typically nonlinear,
  so that the interparticle force changes significantly over small distances.
  %and the simple Euler algorithm is insufficient.
  %[xx we are confused. First you say that Euler works for nonlinear equations, and now you say it is insufficient? suggest omitting this paragraph  xx] Justifiably!  The last phrase is wrong.  Does the paragraph hold if the last phrase was removed, or should the entire paragraph go?
  
\end{problem}

\medskip \begin{problem}{Exploring model parameters}
\label{ex:parameters}

\noindent A range of  behaviors
can be explored by varying the
  relative strength of the confining landscape
  and the external driving force.

\begin{enumerate}

\item[(a)]
  Explore the effect of increasing $F_{\rm ac}$ on the hopping pattern.
  For a single  particle,
  the hopping patterns are typically characterized
  by $n_f$, the number of forward steps,
  versus $n_b$, the number of backward steps  in one
  time period.
  The total displacement of $(n_f - n_b) \lambda$ 
  is the net hop length.  
  In Fig.~\ref{fig:2_Fd_vy_time},
  the particle moves forward through a minima ($n_f = 1$),
  and does not move backward through a full minima ($n_b = 0$).  
  To achieve
  backward hops,
  the  ratio of $F_{\rm ac}/F_{\rm dc}$ must be greater that one  
  and the difference $|F_{\rm ac} - F_{\rm dc}| > A_p$.
  Hold all other parameters fixed and %DM modified typo ``Holdall''
  explore how the hopping pattern
  changes for increasing $F_{\rm ac}$.
  For example,
  compute the patterns of hops for  
  $f=0.01$, $F_{\rm dc}=0.07$, and $A_p = 0.1$
  and
  $F_{\rm ac} = 0.2, 0.3$ and $0.4$.
  
\item[(b)] Explore the effect of increasing the driving frequency on the hopping pattern. In Sec.~\ref{sec:results}
  the frequency is sufficiently low so that the effect of 
  the applied force is large
  over a sustained time interval,
  allowing the particle to hop a substrate maxima.
  For example, fix 
  $F_{\rm dc}=0.1$, $F_{\rm ac}=0.05$, and $A_p = 0.1$,
  and explore
  high frequency 
  ($f = 0.1$),
  intermediate frequency 
  ($f = 0.01$),
  and low frequency
  ($f = 0.005$).
 
\item[(c)] Explore the effect of increasing  $F_{\rm ac}$ on the step pattern in Fig.~\ref{fig:3_sweep_vyFDC}.
  For example, sweep the driving force $F_{\rm dc}$
  over increments of $\Delta F_{\rm dc} = 0.001$
  for    $F_{\rm ac}$ and  $f=0.01$.
  Appropriate values for 
  $F_{\rm ac}$ in the range $[0.1, 0.4]$
  will demonstrate a change in the step width
  in a plot of $\langle v_y \rangle$ versus $F_{\rm dc}$.
  
  \end{enumerate}
  \end{problem}
  
  \begin{problem}{Drag models and Reynolds numbers}
\label{ex:reynolds}

\noindent Stokes' law describes the drag force %xx changed $\vec{F}_{\rm lin}$  to  $\vec{F}_{\rm drag}$     since   you use this notation in Problem 3
  $\vec{F}_{\rm drag} = -3 \pi \eta D \vec{v}$ 
  on a sphere
  moving through a viscous liquid at velocity $\vec{v}$,
  where $\eta$ is the dynamic fluid viscosity and 
  $D$ is the particle diameter.\cite{Taylor2005}
  In simulations we
  subsume the constants $3 \pi D$
  such that $3 \pi D \eta \rightarrow \eta $.
  Often drag forces are
  modeled as a polynomial series\cite{Taylor2005}
$\vec{F}_{\rm drag} = -b \vec{v} - c v^2 \hat{v} + \cdots\,.$
Truncating the series at the first term
  is justified by demonstrating the sphere
  has a low Reynolds number  
  $R = D v \rho / \eta$,
  where $\rho$ is the fluid density and $v$ is the particle's speed.
  If $R$ is small, the quadratic and higher order terms
  may be ignored.

Use reasonable values for the
  experimental analog of this system and show 
that the Reynolds number is small.
In addition to the values listed in Table~\ref{tab:1},
the viscosity is $\eta \sim 10^{-3}$\,Pa-s,\cite{Volpe2013}
and 
the liquid density 
$\rho \sim 10^3$\,kg/cm$^3$.\cite{asce}

\end{problem}


  %-------------------------------------------------------------------------
  
\begin{problem}{Equation of motion}
  \label{ex:n2l}

\noindent Newton's second law states that
  the acceleration of a particle 
  is proportional to 
  the sum of the forces on the particle, $m \vec{a} = \sum \vec{F}$,
  where $m$ is the
  inertial mass.  
  The addition of a dissipative force to a dynamical equation 
  of colloid motion 
  is typically modeled
  by a drag force proportional to the particle's velocity
  in the opposite direction of motion 
  $\vec{F}_{\rm drag} = - \gamma \vec{v}$,
  where $\gamma = 3 \pi \eta D$ is the drag coefficient
  described in Problem~\ref{ex:reynolds}.
  The ratio of $m/\gamma$ is 
 the momentum relaxation time,
  and is small for
  particles with low Reynolds numbers.
  The mass of a
  typical colloid particle is $15$\,picograms,
  leading to a momentum relaxation time
  on the order of microseconds.

\begin{enumerate}

\item [(a)] Given  the values listed in Table~\ref{tab:1},
 show that  the momentum relaxation time is 
  $m/\gamma \approx 0.5\,\mu$s. 
  
\item [(b)] If $m/\gamma$ is small, the
  particle's acceleration can be ignored
  entirely.
Use Newton's second law for
  a small momentum relaxation time and
  show that a particle confined to a landscape exerting force
  $F_\ell(\vec{r})$ subject to a time dependent drive $F_{\rm d}(t)$
  can be modeled by the equation of motion  in 
  Eq.~\eqref{eq:motion}.
  \end{enumerate}
\end{problem}

%--------------------------------------------------------------------------

 \begin{problem}{Brownian motion}
  \label{ex:brownian}
  
\noindent Brownian motion describes the apparent random motion     of 
  visible particles  
  due to collisions with invisible fluid particles.
  The rate of collisions depends on the temperature, viscosity
  and the density of 
  the suspending fluid.\cite{Einstein1905}
  An optically trapped colloid executing Brownian motion
  is a useful probe of microscopic forces.\cite{Volpe2013}
  
  In   simulations
  it is common to treat the 
  invisible fluid particles as a continuous fluid
  to reduce the computational expense.
  Temperature effects
  can be modeled by applying randomized forces $f_T$ 
 to the visible particles ($T$ denotes the temperature).
  We use the normal distribution 
  to generate a series of ${f_T}_n$ values for
  each  time step $n$.\cite{numpy}
  A normalized random distribution of forces
  causes fluctuations in the
  motion 
  equally in all
  directions such that the force $f_T$
  averaged over a finite time interval
  is zero.  In one dimension we have 
$
    \langle f_T(t) \rangle = \frac{1}{N} \sum_n^N {f_T}_n = 0$,
  where the integer $n$ indicates the number of  
    time steps and 
  $t = N \Delta t$.
  A particle
  with sufficient energy $k_B T_{\min}$ may 
  hop over landscape
  barriers.
  In our simulations,
  we let   
  $k_B T/E_0 \rightarrow T$,
  with the constants set to unity
  to
  compare directly with the force. 
 
  For zero applied driving force,
  find 
  the minimum temperature required for a single particle
  to hop over the maxima in the potential landscape.
  Assume that the particle is confined to
  move along the $y$-direction and
  include Brownian motion.
  Appropriate temperatures to explore
  are $3.0 < T/A_p = 7.0$.
Plot the $y$ position versus time
  for this particle.
 

  For a driven particle,
  we ignore 
  the effects of temperature 
  in these simulations.  
  We note that even for $T/A_p = 6.0$, 
  the hopping rate
  is much less than the
  frequency of the applied drive.   
  At sufficiently high temperatures,
  Brownian motion does affect 
  the formation of mode-locked steps
  and can be observed in experiments.
 \end{problem}

 \begin{problem}{Critical Exponents of Depinning}
   \label{ex:critical}
 
   \noindent
   With no oscillating component of the driving force,
  the force-velocity relation increases monotonically 
  above the depinning threshold $F_c$ such that
  \begin{equation}
    \langle v_y \rangle \propto (F_{\rm dc}-F_c)^{-\beta},
  \end{equation}
  where the power $\beta$ varies with the
  type of phase transition and 
  %can
  may 
  be used to identify the universality class.\cite{Reichhardt2017} 
  The critical force $F_c$ is equal to the maximum substrate force
  $A_p$.
   % [xx would be good to follow up on your mention of universality class in a problem. The exponent beta is usually associated with the decrease of the order parameter, not a divergence xx]
  %DM - in my experience, we use beta for this parameter.  For example
  %Dynamic regimes for driven colloidal particles on a periodic substrate at commensurate and incommensurate fillings D McDermott, J Amelang, CJO Reichhardt, C Reichhardt, Physical Review E 88 (6), 062301

  \end{problem}

\begin{acknowledgments}
Charles and Cynthia Reichhardt
  inspired the project and
  provided the original molecular dynamics code
  written in  C.
  We acknowledge funding from the M.J.\ Murdock Charitable Trust
  and the Pacific Research Institute for Science and Mathematics.

\end{acknowledgments}
 
\begin{thebibliography}{99}

  %intro to oscillators
\bibitem{Pikovsky2003} A. Pikovsky, M. Rosenblum, and J. Kurths, {\it Synchronization: A Universal Concept in Nonlinear Sciences} (Cambridge University Press, Cambridge, 2003).
  
\bibitem{Okamoto2016} K. Okamoto, A. Kijima, Y. Umeno, and H. Shima, ``Synchronization in flickering of three-coupled candle flames," Sci. Rep. {\bf 6}, 36145--36155 (2016).

\bibitem{Arane2009} T. Arane, A. K. R. Musalem, and M. Fridman, ``Coupling between two singing wineglasses," Am. J. Phys. {\bf 77}, 1066--1067 (2009). %; https://doi.org/10.1119/1.3119175
  
\bibitem{Jia2015}  J. Jia, Z. Song, W. Liu, J. Kurths, and J. Xiao, ``Experimental study of the triplet synchronization of coupled nonidentical mechanical metronomes," Sci. Rep. {\bf 5}, 17008--17020 (2015).

%Synchronization of a thermoacoustic oscillator by an external sound source
%G. Penelet and T. Biwa
%American Journal of Physics 81, 290 (2013); https://doi.org/10.1119/1.4776189
  %biological examples
  
\bibitem{Portugal2014} S. Portugal, T. Hubel, J. Fritz, S. Heese, D. Trobe, B. Voelkl, S. Hailes, A. M. Wilson, and J. R. Usherwood,   ``Upwash exploitation and downwash avoidance by flap phasing in ibis formation flight," Nature {\bf 505}, 399--402 (2014).

  \bibitem{Aihara2014} I. Aihara, T. Mizumoto, T. Otsuka, H. Awano, K. Nagira, H. G. Okuno, and K. Aihara, ``Spatio-temporal dynamics in collective frog choruses examined by mathematical modeling and field observations," Sci. Rep. {\bf 4}, 3891--3899 (2014). 

  \bibitem{Tranchant2016} P. Tranchant, D. T. Vuvan, and I. Peretz, ``Keeping the beat: A Large sample study of bouncing and clapping to music," PLoS ONE 11(7): e0160178 (2016).

  \bibitem{MartinHall1999} G. Martin Hall, Sonya Bahar, and Daniel J. Gauthier, ``Prevalence of rate-dependent behaviors in cardiac muscle," Phys. Rev. Lett. {\bf 82}, 2995--2998 (1999).

    %huygens clocks
  \bibitem{Singer1999} W. Singer, ``Striving for coherence,"  Nature {\bf 397} 391--393 (1999).

  \bibitem{Bennett2002} M. Bennett, M. F. Schatz, H. Rockwood, and K. Wiesenfeld, ``Huygens' clocks," Proc. Roy. Soc. A {\bf 458}, 563--579 (2002).

    
  %colloids
  \bibitem{Pertsinidis2001} A. Pertsinidis and X. Ling,  ``Equilibrium configurations and energetics of point defects in two-dimensional colloidal crystals," Phys Rev. Lett.  {\bf 87}, 098303-1--4 (2001). %https://doi.org/10.1103/PhysRevLett.87.098303

  \bibitem{Juniper2015} M. P. N. Juniper, A. V. Straube, R. Besseling, D. G. A. L. Aarts, and R. P. A. Dullens, ``Microscopic dynamics of synchronization in driven colloids," Nat. Commun. 6, 7187--7194 (2015). 
      %\bibitem{Juniper2018}
      
  \bibitem{Juniper2017} M. P. N. Juniper,  U. Zimmermann, A. V. Straube, R. Besseling, D. G. A. L. Aarts, H. L{\"o}wen, and R. P. A. Dullens,  ``Dynamic mode locking in a driven colloidal system: Experiments and theory," New J. Phys. {\bf 19} (1), 013010-1--14 (2017).  %https://doi.org/10.1088/1367-2630/aa53cd

  \bibitem{Ashkin1997} A. Ashkin, ``Optical trapping and manipulation of neutral particles using lasers," Proc. Natl. Acad. Sci. U.S.A. {\bf 94}, 4853--4860 (1997).

  \bibitem{Grier2003} D. G. Grier, ``A revolution in optical manipulation," Nature {\bf 424}, 810--816 (2003).

  %general particle on washboard
  \bibitem{Reichhardt2017} C. Reichhardt and C. J. Olson Reichhardt, ``Depinning and nonequilibrium dynamic phases of particle assemblies driven over random and ordered substrates: a review," Rep. Prog. Phys. {\bf 80}, 026501-1--64 (2017). %[I don't know what you mean.  64?  xx \# of pages? xx - https://iopscience.iop.org/article/10.1088/1361-6633/80/2/026501/meta]

  \bibitem{Reichhardt2015} C. Reichhardt, and C. J. O. Reichhardt,  ``Shapiro steps for skyrmion motion on a washboard potential with longitudinal and transverse ac drives," Phys. Rev. B {\bf 92} (22), 224432-1--12 (2015).      

  %overdamped justification
  \bibitem{Purcell1977} E. M. Purcell, ``Life at low Reynolds numbers,"  Am. J. Phys. {\bf 45}, 3--11 (1977).
  
  %chaos!
  \bibitem{Bak1986} P. Bak. ``The Devil's staircase," Physics Today {\bf 39}(12), 38--45 (1986).

  %overdamped justification
  \bibitem{Taylor2005} J. Taylor,  {\it Classical Mechanics} (University Science Books, 2005).

  %brownian motion!
  \bibitem{Volpe2013} G. Volpe and G. Volpe, ``Simulation of a Brownian particle in an optical trap,"  Am. J. Phys. {\bf 81}(3), 224--230 (2013).

  %viscosity and density citation
  \bibitem{asce}
    %Non-fundamental constants for water viscosity are available
    %in several databases including 
    %  (1) https://ascelibrary.org/doi/pdf/10.1061/9780784408230.ap02
    IAPWS R12-08, 
    ``Release on the IAPWS formulation 2008 for the viscosity of ordinary water substance,  September 2008,
    %IAPWS 2008 -
    \url{<http://www.iapws.org/relguide/viscosity.html>}.
    %[citation for these values better than wikipedia] \cite{} %CONFIRM!
    %density $\rho = 0.9982 g/cm^3$

  %numerical solutions of differential equations
  \bibitem{Newman} M. Newman, {\it Computational Physics} (CreateSpace Independent Publishing Platform, 2012).

  \bibitem{supplemental}
    %not sure if the manuscript can remain in the repo - that probably violate AIP rules https://publishing.aip.org/resources/researchers/rights-and-permissions/author-licenses/
    %will make a DOI for the repo
    %https://guides.github.com/activities/citable-code/
    \url{<https://github.com/mcddanielle/AJP_project/>}.

    %Brownian motion!
  \bibitem{Einstein1905} A. Einstein, {\it Investigations on the Theory of the Brownian Movement}  (Dover Publications, 1956).

    \bibitem{numpy} C. R. Harris, K. J. Millman, S. J. van der Walt et al.,  ``Array programming with NumPy,"  Nature {\bf 585}, 357--362 (2020). %DOI: 0.1038/s41586-020-2649-2. (Publisher link).
      
   %Josephson Junctions and other condensed matter applications
    \bibitem{Shapiro1963} S. Shapiro, ``Josephson currents in superconducting tunneling: The effect of microwaves and other observations," Phys. Rev. Lett. {\bf 11}, 80--82 (1963).

    \bibitem{Golubov2004} A. A. Golubov, M. Yu. Kupriyanov, and E. Il{\`i}chev, ``The current-phase relation in Josephson junctions," Rev. Mod. Phys. {\bf 76}, 411--469 (2004).

    \bibitem{Benz1990}  S. P. Benz, M. S. Rzchowski, M. Tinkham, and C. J. Lobb, ``Fractional giant Shapiro steps and spatially correlated phase motion in 2D Josephson arrays," Phys. Rev. Lett. {\bf 64}, 693--696 (1990); D. Dom{\'i}nguez and J. V. Jos{\'e}, ``Giant Shapiro steps with screening currents," Phys. Rev. Lett. {\bf 69}, 514--517 (1992).

    \bibitem{Reichhardt2000} C. Reichhardt, R. T. Scalettar, G. T. Zim{\'a}nyi, and N. Gr{\o}nbech-Jensen,  ``Phase-locking of vortex lattices interacting with periodic pinning,"  Phys. Rev. B {\bf 61}, R11914 (2000).
     

\end{thebibliography}

\end{document}

